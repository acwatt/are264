\documentclass[12pt]{article}
\usepackage{preamble}
\graphicspath{{pics/PS1/}}
\begin{document}
% \maketitle
\chead{Problem Set 1}

%%%%%%%%%%%%%%%%%%%%%%%%%%%%%%%%%%%%%%%%%%%%%%%
%                  Definitions                %
%%%%%%%%%%%%%%%%%%%%%%%%%%%%%%%%%%%%%%%%%%%%%%%
%\includegraphics[width=\mywidth\textwidth]{}

% \begin{figure}[h!]
% \centering
% \input{pics/PS2/p7b}
% \caption{}
% \label{fig-}
% \end{figure}

% \begin{enumerate}[label=\alph*.]
%     \setcounter{enumi}{1}
%     \item 
% \end{enumerate}

%%%%%%%%%%%%%%%%%
%     Part a    %
%%%%%%%%%%%%%%%%%


%%%%%%%%%%%%%%%%%%%%%%%%%%%%%%%%%%%%%%%%%%%%%%%
%                Problem 1                    %
%%%%%%%%%%%%%%%%%%%%%%%%%%%%%%%%%%%%%%%%%%%%%%%
\section*{Problem 1}
\problem{Clean subsidies versus dirty taxes}{
    Many environmental policies take the form of subsidies for clean goods or activities, rather than
taxes on the dirty good or activity. Presumably such instruments are popular in part because it is
more politically appealing to give away a subsidy than to collect a tax. The aim of this exercise is
to get you to consider the economic efficiency of using “green subsidies” in lieu of “brown taxes”
(or “clean subsidies” in lieu of “dirty taxes”).\\

Consider the following model (which we used in lecture). There P are heterogeneous producers indexed 
$j = 1, ..., J$, which produce homogeneous good $x_j ; X \equiv\sum_{j=1}^J x_j$ , according to
production cost $c_j(x_j )$ (with $c_j' > 0, c_j'' > 0$). A representative consumer has utility U (X) (with U 0 > 0,
U 00 ≤ 0). Production creates an externality according to e j (x j ), (with e 0 j > 0, e 00 j ≤ 0). Firms
can abate emission a j , at cost g j (a j ) (with g j 0 > 0, g 00 > 0). Net emissions from firm j are thus
P
e j (x j ) − a j , and total damages to the consumer are φ Jj=1 (e j (x j ) − a j ).
}

\begin{align} 
    \intertext{stuff}
    y&=x
\end{align}


%%%%%%%%%%%%%%%%%%%%%%%%%%%%%%%%%%%%%%%%%%%%%%%
%                Problem 2                    %
%%%%%%%%%%%%%%%%%%%%%%%%%%%%%%%%%%%%%%%%%%%%%%%
\newpage
\section*{Problem 2}
\problem{Problem name}{
    Problem description
}
\begin{align} 
    \intertext{stuff}
    y&=x
\end{align}



%%%%%%%%%%%%%%%%%%%%%%%%%%%%%%%%%%%%%%%%%%%%%%%
%                Problem 3                    %
%%%%%%%%%%%%%%%%%%%%%%%%%%%%%%%%%%%%%%%%%%%%%%%
\newpage
\section*{Problem 3}
\problem{Problem name}{
    Problem description
}

\begin{align} 
    \intertext{stuff}
    y&=x
\end{align}


%%%%%%%%%%%%%%%%%%%%%%%%%%%%%%%%%%%%%%%%%%%%%%%
%                Problem 4                    %
%%%%%%%%%%%%%%%%%%%%%%%%%%%%%%%%%%%%%%%%%%%%%%%
\newpage
\section*{Problem 4}
\problem{Problem name}{
    Problem description
}

\begin{align} 
    \intertext{stuff}
    y&=x
\end{align}



%%%%%%%%%%%%%%%%%%%%%%%%%%%%%%%%%%%%%%%%%%%%%%%
%                Problem 5                    %
%%%%%%%%%%%%%%%%%%%%%%%%%%%%%%%%%%%%%%%%%%%%%%%
\newpage
\section*{Problem 5}
\problem{Problem name}{
    Problem description
}

\begin{align} 
    \intertext{stuff}
    y&=x
\end{align}


%%%%%%%%%%%%%%%%%%%%%%%%%%%%%%%%%%%%%%%%%%%%%%%
%                Problem 6                    %
%%%%%%%%%%%%%%%%%%%%%%%%%%%%%%%%%%%%%%%%%%%%%%%
\section*{Problem 6}
\problem{Problem name}{
    Problem description
}

\begin{align} 
    \intertext{stuff}
    y&=x
\end{align}



%%%%%%%%%%%%%%%%%%%%%%%%%%%%%%%%%%%%%%%%%%%%%%%
%                Problem 7                    %
%%%%%%%%%%%%%%%%%%%%%%%%%%%%%%%%%%%%%%%%%%%%%%%
\section*{Problem 7}
\problem{Problem name}{
    Problem description
}

\begin{align} 
    \intertext{stuff}
    y&=x
\end{align}


%%%%%%%%%%%%%%%%%%%%%%%%%%%%%%%%%%%%%%%%%%%%%%%
%                Problem 8                    %
%%%%%%%%%%%%%%%%%%%%%%%%%%%%%%%%%%%%%%%%%%%%%%%
\section*{Problem 8}
\problem{Problem name}{
    Problem description
}

\begin{align} 
    \intertext{stuff}
    y&=x
\end{align}



%%%%%%%%%%%%%%%%%%%%%%%%%%%%%%%%%%%%%%%%%%%%%%%
%                Problem 9                    %
%%%%%%%%%%%%%%%%%%%%%%%%%%%%%%%%%%%%%%%%%%%%%%%
\section*{Problem 9}
\problem{Problem name}{
    Problem description
}

\begin{align} 
    \intertext{stuff}
    y&=x
\end{align}


%%%%%%%%%%%%%%%%%%%%%%%%%%%%%%%%%%%%%%%%%%%%%%%
%                Problem 10                   %
%%%%%%%%%%%%%%%%%%%%%%%%%%%%%%%%%%%%%%%%%%%%%%%
\section*{Problem 10}
\problem{Problem name}{
    Problem description
}

\begin{align} 
    \intertext{stuff}
    y&=x
\end{align}




\end{document}

