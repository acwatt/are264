\documentclass[12pt]{article}
\usepackage{../preamble}
\graphicspath{{pics/hw1/}}
\begin{document}
% \maketitle
\chead{Problem Set 1}

%%%%%%%%%%%%%%%%%%%%%%%%%%%%%%%%%%%%%%%%%%%%%%%
%                  Definitions                %
%%%%%%%%%%%%%%%%%%%%%%%%%%%%%%%%%%%%%%%%%%%%%%%
%\includegraphics[width=\mywidth\textwidth]{}

% \begin{figure}[h!]
% \centering
% \input{pics/PS2/p7b}
% \caption{}
% \label{fig-}
% \end{figure}

% \begin{enumerate}[label=\alph*.]
%     \setcounter{enumi}{1}
%     \item 
% \end{enumerate}

%%%%%%%%%%%%%%%%%
%     Part a    %
%%%%%%%%%%%%%%%%%


%%%%%%%%%%%%%%%%%%%%%%%%%%%%%%%%%%%%%%%%%%%%%%%
%                Problem 1                    %
%%%%%%%%%%%%%%%%%%%%%%%%%%%%%%%%%%%%%%%%%%%%%%%
\section*{Problem 1}
\problem{Clean subsidies versus dirty taxes}{
    Many environmental policies take the form of subsidies for clean goods or activities, rather than taxes on the dirty good or activity. Presumably such instruments are popular in part because it is more politically appealing to give away a subsidy than to collect a tax. The aim of this exercise is to get you to consider the economic efficiency of using “green subsidies” in lieu of “brown taxes” (or “clean subsidies” in lieu of “dirty taxes”).\\

Consider the following model (which we used in lecture). There P are heterogeneous producers indexed $j = 1, ..., J$, which produce homogeneous good $x_j ; X \equiv\sum_{j=1}^J x_j$ , according to production cost $c_j(x_j )$ (with $c_j' > 0, c_j'' > 0$). A representative consumer has utility $U(X)$ (with $U' > 0$, $U'' \leq 0$). Production creates an externality according to $e_j(x_j)$, (with $e_j' > 0$, $e_j'' \leq 0$). Firms can abate emission $a_ j$, at cost $g_j(a_j)$ (with $g_j' > 0$, $g_j'' > 0$). Net emissions from firm $j$ are thus $e_j(x_j) - a_j$, and total damages to the consumer are $\phi\sum_{j=1}^J (e_j(x_j) - a_j)$.\\

Suppose that the only instrument the planner has available is a subsidy, denoted $s$, which can be awarded to each firm based on its level of abatement $a_j$ , for a total subsidy of $sa_j$ . The planner can choose the level of the subsidy, but no other policies are available.\\

\textbf{Evaluate the following two claims; that is, prove their veracity or falsity using the model:}
\begin{enumerate}
    \item An abatement subsidy can be used to decentralize the optimal allocation.
    \item An abatement subsidy can be used to generate the socially optimal level of abatement at each firm.
\end{enumerate}
}


\def\sumj{\sum\limits_j}
\def\cj{c_j(x_j)} \def\cjprime{c_j'(x_j)}
\def\gj{g_j(a_j)} \def\gjprime{g_j'(a_j)}
\def\ej{e_j(x_j)} \def\ejprime{e_j'(x_j)}
\def\dda{\frac{\partial}{\partial a_j}} \def\ddx{\frac{\partial}{\partial x_j}}
\begin{align*} 
\intertext{Let $A\equiv\sum_j a_j$ and $E\equiv\sum_j e_j(x_j)$. To find the optimal allocation, consider the social planner's problem:}
    \max_{a_j,x_j} SWF &= \text{Benefit of consuming $X$} - \text{Cost of producing $X$}
        - \text{Cost of abating $A$} - \text{Cost of externality E}\\
    &= U(X) - \sumj(\cj +\gj) - \phi\sumj(\ej - a_j)
\intertext{The First Order Conditions describing the optimal allocation and firm-level abatement are then}
    \dda SWF &= \phi - \gjprime = 0 \implies \gjprime = \phi\\
    \ddx SWF &= \ddx U - \cjprime - \phi\cdot\ejprime = 0 \implies U'(X) = \cjprime + \phi\cdot\ejprime
\intertext{Next, to see the effect of the subsidy, let $P$ be the market price of the homogeneous good and consider the firm $j$'s profit maximization problem:}
    \max_{a_j,x_j} \pi_j &= \text{Revenue from $x_j$} - \text{Cost of producing $x_j$}
        - \text{Cost of abating $a_j$} + \text{subsidy from abating $a_j$}\\
    &= P\cdot x_j - \cj - \gj + s\cdot a_j
\intertext{The First Order Conditions for profit maximization are then}
    \dda\pi_j &= -\gjprime + s = 0 \implies \gjprime = s\\
    \ddx\pi_j &= P - \cjprime = 0 \implies \cjprime = P
\end{align*}

\subsection{Can an abatement subsidy can be used to decentralize the optimal allocation?}

\def\xs{x_j^{s}} \def\xsp{x_j^{SP}}
\begin{align*}
\intertext{Let $\{\xsp\}_j$ be the optimal allocation resulting from the solution to the social planner's problem, and let $\{\xs\}_j$ be the allocation resulting from the subsidy level $s$. Suppose that, for some subsidy level $s$, the abatement subsidy achieved the optimal allocation. That is, assume that $\xs = \xsp \ \forall j$. Then the corresponding aggregate goods would be equal: $X^s = X^{SP}$}
\intertext{Then, by the FOCs above,}
    c_j'(x_j^s) = U'(X^s) \\
        &= U'(X^{SP}) \\
        &= c_j'(x_j^{SP}) + \phi\cdot e_j'(x_j^{SP}) \\
        &= c_j'(x_j^s) + \phi\cdot e_j'(x_j^s)
\intertext{Assuming the damages from the externality are nonzero (i.e., $\phi\neq 0$), and since $e_j'>0$, }
    c_j'(x_j^s) &\neq c_j'(x_j^s) + \phi\cdot e_j'(x_j^s)
\intertext{This is a contradiction. Therefore,}
    \exists j &\st x_j^s \neq x_j^{SP}
\intertext{So the allocation under the subsidy cannot be the optimal allocation -- the subsidy cannot be used to decentralize the optimal allocation.}
\end{align*}


\subsection{Can an abatement subsidy can be used to generate the socially optimal level of abatement at each firm?}

\begin{align*}
\intertext{From the social planner's FOCs, we have}
    \gjprime &= \phi
\intertext{and from the firm's profit maximization FOC under the subsidy, we have}
    \gjprime &= s
\intertext{So if we set $s=\phi$, the firms will abate until their marginal cost of abatement equals the marginal damage of the externality. Indeed, the subsidy can be used to generate the socially optimal level of abatement at each firm if the marginal damage of the externality $\phi$ is known.}
\end{align*}



%%%%%%%%%%%%%%%%%%%%%%%%%%%%%%%%%%%%%%%%%%%%%%%
%                Problem 2                    %
%%%%%%%%%%%%%%%%%%%%%%%%%%%%%%%%%%%%%%%%%%%%%%%
\newpage
\section*{Problem 2}
\problem{Variance and value}{
    The first-best Pigouvian tax level for a good that causes an externality does not depend on the elasticity of supply and demand. But, the welfare gain achieved by the implementation of the tax does.\\

Similarly, the first-best Pigouvian tax does not depend on the heterogeneity in the cost of abatement opportunities. But, the relative efficiency of a tax versus a command-and-control alternative does depend on the degree of heterogeneity.\\

\textbf{Write down a deliberately simple model that can be used to illustrate these two facts (specifically, that welfare gains from implementing an optimal tax and gains relative to command and control depend on variance), and then illustrate them.}\\

The ideal result would produce an expression that describes the welfare gain from efficiency as a result of the variance (or other measure of dispersion) in marginal costs.\\

You might consider using some version of the abatement cost model from class and from the problem above, but you can likely simplify further by reducing that model to have only one margin of adjustment (i.e., each firm has a constant quantity of output, but can choose abatement levels).\\

Do consider drawing trying to write down a model that will allow you to graph your result.
}
\begin{align} 
    \intertext{stuff}
    y&=x
\end{align}



%%%%%%%%%%%%%%%%%%%%%%%%%%%%%%%%%%%%%%%%%%%%%%%
%                Problem 3                    %
%%%%%%%%%%%%%%%%%%%%%%%%%%%%%%%%%%%%%%%%%%%%%%%
\newpage
\section*{Problem 3}
\problem{Two related dirty goods}{
    You wish to comment on the optimal tax for a good that creates a negative externality, but which is a relatively clean version of a product. That is, close substitutes for the good create an even larger externality. (I.e., you are thinking about taxing natural gas relative to coal; or hybrid cars relative to conventional vehicles; or energy-efficient appliances relative to less efficient ones.) Others debating this policy believe that the relatively clean product should be subsidized (not taxed) because the dirtier products are undertaxed relative to their marginal damages. You are not sure if the product should be subsidized, so you wish to describe the optimal tax on the relatively clean good, under the assumption that the dirtier substitute is undertaxed.\\
    
    \textbf{Write down a mathematical model that allows you to describe the optimal tax on the relatively clean good, assuming that the tax on the dirtier good is fixed and you cannot change it. Solve the model and describe how the results answer the broader question of how mispriced substitutes can affect optimal policy design for a dirty good.}
}

\begin{align} 
    \intertext{stuff}
    y&=x
\end{align}

 
\end{document}

